% !TeX root = Notes.tex

\documentclass{article}
\usepackage{ctex}
\usepackage[utf8]{inputenc}
\usepackage[top=1in, bottom=1in, left=1.in, right=1.in]{geometry} % adjust default margins
\usepackage{fancyhdr}
\pagestyle{fancy}
\lhead{Yiran Wang}\chead{\reportNumber}\rhead{Rutgers Univ.}\lfoot{}\cfoot{\thepage}\rfoot{} % setting page header and footer
\setlength{\parindent}{0em}\setlength{\parskip}{1.5ex} % use newline to seperate paragraphs
\usepackage[T1]{fontenc}     % oriented to output, that is, what fonts to use for printing characters % [must use] make ligatures copyable
\usepackage{enumitem} %\setlist[itemize]{noitemsep,nolistsep} % same to use \begin{itemize}[nolistsep,noitemsep]
\usepackage[ampersand]{easylist} % ampersand == & % \begin{easylist}[itemize] %% to make easy list environment
\usepackage{amsmath,amssymb,bm}
\usepackage[colorlinks=true,linkcolor=blue,citecolor=blue,urlcolor=blue,breaklinks=true]{hyperref} % color the hyperlink rather than box it
\usepackage{graphicx}
\usepackage{color}
\usepackage[all]{hypcap} % make the hyperlink to point at the upper border of a figure of table, rather than to point to its caption.
\usepackage{breakcites}
\usepackage{cleveref} \crefname{equation}{Eq.}{Eqs.} \Crefname{equation}{Equation}{Equations} \crefname{figure}{Fig.}{Figs.} \Crefname{figure}{Figure}{Figures} \crefname{table}{Table}{Tables} % allow refer to multiple equations and change cross-reference name
\usepackage{caption}
\usepackage{subfigure}
\usepackage{wasysym} % to insert emotion \smiley
\usepackage{float} 
\usepackage[yyyymmdd,hhmmss]{datetime}\renewcommand{\dateseparator}{-}
\usepackage{placeins}
%\usepackage{setspace} \doublespacing \onehalfspacing % 调整行距
%\usepackage{pdflscape} \horizontal figure
\usepackage{ulem}
\usepackage{mathtools}
\urlstyle{same}
%\graphicspath{ {./Test_01/} }
\bibliographystyle{abbrv}
%%%%%%%%%%%%%%%%%%%%%%%%%%%%%%%%%%%%%%%%%%%%%%%%%%%%%%%
\newcommand{\phSay}[1]{\textcolor{blue}{#1}}
%%%%%%%%%%%%%%%%%%%%%%%%%%%%%%%%%%%%%%%%%%%%%%%%%%%%%%%

\title{Ads Notes}
\author{\mathfrak{Y} \mathfrak{R} }
\date{Updated on \today}

\begin{document}
\maketitle
\section{OCPX}
广告变现:本质是“用户流量换取广告主付费”。通过精准触达目标用户,实现转化并收取广告费。

OCPX: Optimized Cost Per X. 为了让广告系统持续优化转化效果,把用户的转化数据回传给广告平台。

CPX: Cost per X, X can be click, install, purchase, lead, etc. CPC 点击。 CPA 安装。 CPI 购买。 CPL 潜在客户。

oCPX: Optimized Cost per X. 智能出价模式。通过机器学习算法,自动优化广告投放,提升转化率。以更低成本获得更多转化。

\begin{itemize}
    \item CPM: Cost per Mille, 每千次展示费用。按曝光量计费。
    \begin{itemize}
        \item 优:适合品牌曝光,快速触达大规模用户。计算逻辑简单,便于广告主预算规划
        \item 缺:无法保证点击和转化,效果难以量化。易被工具屏蔽。
    \end{itemize}

    \item CPC: Cost per Click, 每次点击费用。按点击量计费。典型例子:搜索广告(百度,Google),信息流广告(今日头条,Facebook)
    \begin{itemize}
        \item 优:按点击付费,风险较低。适合引导用户访问网站或App。广告主只为兴趣买单,ROI可控,适合明确转化目标
        \item 缺:点击不等于转化,可能存在无效点击。对广告创意和落地页要求高,需要精确匹配受众。易被恶意点击
    \end{itemize}

    \item 转化广告:
    
    CPA (Action):按完成特定动作计算(注册、下载、购买等)。典型例子:电商平台(淘宝,亚马逊),应用商店(App Store,Google Play)
    
    CPS(Sale): 按销售额的一定比例计费。适合电商平台,风险低,效果可量化

    CPL(Lead):按获取潜在客户计费(用户提交表单或留下联系方式)。适合B2B业务,风险低,效果可量化
    \begin{itemize}
        \item 优:导向强,风险小,可通过算法优化。适合有明确转化目标的广告主。ROI高,预算利用率高
        \item 缺:需要精确归因能力,初期样本不足,转换慢,对广告创意和落地页要求高
    \end{itemize}

    \item 原生广告 + 内容营销(小红书)
    \begin{itemize}
        \item 优: 用户接受度高,干扰度低,对品牌故事产品口碑建设有效。
        \item 缺: 内容成本高,转化链路长,效果衡量复杂
    \end{itemize}

    \item  程序化广告 + RTB(Real Time Bidding 实时竞价)
    \begin{itemize}
        \item 优:精准投放,按用户画像和行为实时出价,提高广告库存利用率
        \item 缺:技术和数据依赖度高,数据隐私可能影响可用性
    \end{itemize}

\end{itemize}



\section{回传}

\subsection{回传策略}
\begin{itemize}
    \item 哪些事件:
    \begin{itemize}
        \item 重要事件:购买、注册、下单、支付、激活等
        \item 次要事件:加购、浏览、搜索等
        \item 主次转化:eg, 以“付费”为目标, 或“注册”为辅助信号
        \end{itemize}
    \item 时间策略:
    \begin{itemize}
        \item 实时: 发生行为立即上报,利于模型调整
        \item 延迟:反作弊或隐私保护
        \item 批量:定时把一定时间内的数据打包上报
    \end{itemize}
\end{itemize}

\subsection{回传方式}
\begin{itemize}
    \item 客户端SDK集成:在App或网站中集成广告平台的SDK,自动收集和上报数据
    \item API接口(server-to-server: S2S):通过调用广告平台提供的API接口,手动上报。最稳定,适合跨平台,app端
    \item 离线文件(eg:门店消费):定期生成包含转化数据的文件,上传到广告平台
\end{itemize}

\subsection{回传数据质量要求}
\begin{itemize}
    \item 唯一事件ID:每个转化事件需有唯一标识,避免重复计数
    \item 时间戳:准确记录事件发生时间,便于归因和分析
    \item 事件类型:明确区分不同类型的转化事件
    \item 用户匹配参数:如IDFA、GAID、手机号等,确保能匹配到正确用户
    \item 完整性:确保所有重要事件都被回传,避免数据遗漏
    \item 及时性:数据应尽可能实时回传,减少延迟
\end{itemize}

\subsection{频率与比例}
\begin{itemize}
    \item 只传部分事件(防止过拟合),或只传高质量转化(高客单价付费)
    \item “漏斗回传”:只传最终转化事件,或传主要+次要事件。先注册,再付费。
\end{itemize}

\subsection{流程}
\begin{enumerate}
    \item 起量期: 上传优化,保证足够多样本量,优先积累数据。
    \item 稳定期:切换到核心事件(付费,留存)。保留部分辅助事件(注册,浏览)回传。可适度延迟,减少作弊风险。
    \item 防作弊期:延迟回传高价值转化,S2S回传+数据脱敏
    \item 多媒体投放:不同媒体设置差异化策略
\end{enumerate}


\section{评价广告差评的核心竞争力}
\begin{itemize}
    \item 流量与用户的价值
    \item 广告效果与转化能力 -- 市场
    \item 产品矩阵与广告形态 -- 产品
    \item 平台能力与运营服务 --商业
    \item 商业模式与价格策略
    \item 合规性与行业趋势
\end{itemize}

高效投放策略: 

目标(Why) -- 数据(What,Where) -- 策略(What, How) -- 执行(When) -- 优化(track)

ads network: 流量批发商,把多个媒体的广告位整合起来,打包卖给广告主或代理商。会对广告库做分类(行业,地域,受众),方便购买

SSP: supply side platform.  供给方平台。帮助媒体卖广告位。管理和优化广告库存,连接多个广告网络和DSP

ADX: Ad Exchange. 广告交易平台。实时竞价买卖广告位。连接SSP和DSP

聚合广告: Ad mediation. 通过一个平台管理多个广告网络。优化收益和填充率
 
流程:

媒体(APP/网站) -- SSP/ADX -- DSP/广告网络 -- 广告主/代理商


\section{广告收入计算}

\subsection{CPM: Cost Per Mile. 千次曝光量计费}


\begin{equation}
        \text{Actual Revenue} 
    = ( PV \times \text{Fill Rate} \times (1 - \text{Discount Rate})) 
      \times CPM \times 100\%
    \label{eq:CPM}
\end{equation}

 实际收入 $= (PV \times$ 填充率 $\times$ (1 - 折损率)) $\times$ $CPM * 100\% $

 理解: 
 
 100次广告请求中80次成功展示,填充率(Fill  Rate)为80\%.
 
 折损率:广告加载失败(技术原因为主)约 5-10\%.
 
 eg: 某app广告位每天有100万次广告请求。 填充率70\%, 折损率10\%. CPM 单位5.

 实际展示量 $100,0000 \times 70\% \times (1 - 10\%) = 63,0000$ (次).

 收入: $ (63,0000 / 1000) \times 5 = 3150$ 
 

 \subsection{CPC: Cost Per Click. 点击率付费}
 实际收入 $= (PV \times$ 填充率 $\times$ CTR $\times$ (1 - 折损率)) $\times$ $CPC * 100\% $

填充率$\downarrow \rightarrow$ 可展示广告量 $\downarrow \rightarrow$ 总点击量 $\downarrow \rightarrow$ 收入 $\downarrow$ 

eg:

信息流广告日均50万次请求。填充率$80\%$,折损率$8\%$, CTR $\%2$, CPC 单价0.3.

有效展示量: 50,0000 $\times$ $80\%$ $\times$ (1 - 8\%) = 368, 000

点击量: 368,000 $\times 2\%$  = 7360

收入: 7360 $\times$ 0.3 = 2208


\subsection{CPA: Cost Per Action}
 实际收入 $= (PV \times$ 填充率 $\times$ CTR $\times$ CVR $\times$(1 - 折损率)) $\times$ $CPA * 100\% $

 CVR: 转化率

eg:
200万次广告请求。填充率$65\%$,折损率$12\%$, CTR $1.5\%$, CVR $5\%$, CPA 单价10.

有效展示量: 200,0000 $\times 60\% \times (1-12\%) = 1,144,000$

点击量: 1,144,000 $\times 1.5\% = 17160$

转化量: 17160 $ \times 5\% = 858$

收入: 858 $\times$ 10 = 8580


\subsection{CPT: Cost per Time. 按时间付费}
填充率与折损率由媒体承担(合约保量),广告主按固定时间付费,但媒体需要保证填充率达标。

媒体侧隐形成本: 若填充率不足,会浪费剩余流量或低价填充。

eg:

广告合约保证100万次展示。 实际填充率$90\%$, 折损率$10\%$, CPT单价8000/天。补偿条款:未达展示量按CPM=3补足。

实际展示量: $100,0000 \times 90\% \times (1-10\%) = 810,000$

差额量: 100,0000 - 810,000 = 190,000 次 

补偿成本: $(190,000 / 1000) * 3 = 570$ 

媒体最终收入: 8000 - 570 = 7430

\subsection{广告竞价 eCPM: 预估千次展示收益}

eCPM = 商家出价  $\times$ 预估点击率 CTR $\times$ 预估转化率 CVR $\times$ 1000

 $eCPM \varpropto \text{Price} \times CTR \times CVR$ 

平台根据eCPM高低排序,决定广告展示顺序。

% \bibliography{reference}
\end{document}