% !TeX root = Notes.tex

\documentclass{article}
\usepackage{ctex}
\usepackage[utf8]{inputenc}
\usepackage[top=1in, bottom=1in, left=1.in, right=1.in]{geometry} % adjust default margins
\usepackage{fancyhdr}
\pagestyle{fancy}
\lhead{Yiran Wang}\chead{\reportNumber}\rhead{DayDayUp~}\lfoot{}\cfoot{\thepage}\rfoot{} % setting page header and footer
\setlength{\parindent}{0em}\setlength{\parskip}{1.5ex} % use newline to seperate paragraphs
\usepackage[T1]{fontenc}     % oriented to output, that is, what fonts to use for printing characters % [must use] make ligatures copyable
\usepackage{enumitem} %\setlist[itemize]{noitemsep,nolistsep} % same to use \begin{itemize}[nolistsep,noitemsep]
\usepackage[ampersand]{easylist} % ampersand == & % \begin{easylist}[itemize] %% to make easy list environment
\usepackage{amsmath,amssymb,bm}
\usepackage[colorlinks=true,linkcolor=blue,citecolor=blue,urlcolor=blue,breaklinks=true]{hyperref} % color the hyperlink rather than box it
\usepackage{graphicx}
\usepackage{color}
\usepackage[all]{hypcap} % make the hyperlink to point at the upper border of a figure of table, rather than to point to its caption.
\usepackage{breakcites}
\usepackage{cleveref} \crefname{equation}{Eq.}{Eqs.} \Crefname{equation}{Equation}{Equations} \crefname{figure}{Fig.}{Figs.} \Crefname{figure}{Figure}{Figures} \crefname{table}{Table}{Tables} % allow refer to multiple equations and change cross-reference name
\usepackage{caption}
\usepackage{subfigure}
\usepackage{wasysym} % to insert emotion \smiley
\usepackage{float} 
\usepackage[yyyymmdd,hhmmss]{datetime}\renewcommand{\dateseparator}{-}
\usepackage{placeins}
%\usepackage{setspace} \doublespacing \onehalfspacing % 调整行距
%\usepackage{pdflscape} \horizontal figure
\usepackage{ulem}
\usepackage{mathtools}
\usepackage{longtable}
\urlstyle{same}
%\graphicspath{ {./Test_01/} }
\bibliographystyle{abbrv}
%%%%%%%%%%%%%%%%%%%%%%%%%%%%%%%%%%%%%%%%%%%%%%%%%%%%%%%
\newcommand{\yrSay}[1]{\textcolor{blue}{#1}}
%%%%%%%%%%%%%%%%%%%%%%%%%%%%%%%%%%%%%%%%%%%%%%%%%%%%%%%

\title{LLM Notes}
\author{\mathfrak{Y} \mathfrak{R} }
\date{Updated on \today}

\begin{document}
\maketitle
一些专业名词:


    
    \begin{longtable}{p{3cm}p{5cm}p{9cm}}\toprule
    \hline
中文分词 & Chinese Word Segmentation, CWS & 处理中文文本时,由于词与词之间灭有明显分隔(空格),所以无法直接通过空格来确定词的边界。
                                            其目的是将连续的中文文本切分成有意义的词汇序列。 \\
\hline
子词切分 & Subword Segmentation           &   特别适合处理词汇稀疏的问题,即当遇到罕见词或者未见过的新词时,能够通过已知的子词单位来理解或生成这些词汇。
                                            在处理拼写复杂,合成词多的语言(德语)或预训练语言模型(BERT,GPT)中尤为重要。
                                            常用的方法有Byte Pair Encoding (BPE), WordPiece, Unigram, SentencePiece。\\                                                                 \\
\hline
词性标注 & Part of speech Tagging, POS Tagging & 为文本中的每一个单词分配一个词性标签,如名词动词形容词。 
                                                POS tagging对理解句子结构,进行句法分析,语义角色标注等高级NLP任务至关重要。
                                                计算机可以更好地理解文本的含义,进行信息提取,情感飞,机器翻译。。
                                                通常依赖于机器学习模型,如隐马尔科夫模型(Hidden Markov Model HMM), 条件随机场(COnditional Random Field CRF),或RNN,LSTM等。
                                                通过学习大量的标注数据来预测新句子中每个单词的词性。\\
\hline
文本分类 & Text Classfication & 将给定的文本自动分配到一个或多个预定义的类别中。应用包括但不限于情感分析,垃圾邮件检测,新闻分类,主题识别等。
                                文本分类的关键在于理解文本的含义和上下文,并基于此将文本映射到特定的类别。
                                文本分类的关键在于选择合适的特征表示和分类算法,以及拥有高质量的训练数据。\\
\hline
 实体识别 (又名,命名实体识别) & Named Entity Recognition, NER & 自动识别文本中具有特定意义的实体,并将它们分类为预定的类别,如人名,地点,组织,日期,时间等。
                                            实体识别任务对于信息提取,知识图谱构建,问答系统,内容推荐等应用很重要,它能够帮助系统理解文本中的关键元素及其属性。\\

 \hline

\hline  
\caption{LLM}
\end{longtable}


% \bibliography{reference}
\end{document}